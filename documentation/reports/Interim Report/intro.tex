\section{Introduction}
This project focuses on the organisation and display of sheet music. Sheet music refers only to the instructions given to a performer in order to play a composition, and does not include the sound output produced when the piece is performed. Any reference to music from this point onward should be assumed to mean visual sheet music, rather than audio recordings.
Whilst Eastern countries and previous eras have used different methods of notation \parencite{Kaufman}, this project focuses solely on the notation used by western classical music. This type of musical notation is the format most commonly used by orchestras and performers, and in order to understand the problem of organising it, some explanation of the key elements will be required.

Notation of western classical music has used a combination of diastematic and orthographic notation \parencite{RRastall} since the advent of Gregorian chant around 640AD \parencite{RTaruskin}. The function of this notation falls into two main divisions: the expression of relationship in sound frequency, and the expression of relationship in time, or measure \parencite{oxHistory}.

The representations of other parameters in staff notation are normally orthographic, such as indications p - meaning piano, or "quiet" - and pizz - meaning pizzicato, or "plucked" \parencite{RRastall}. This mechanism is complex in nature, and has a large but finite set of symbols which control every element of the composition. This will be discussed and explained later in this report.

Despite the advantages of digitising music collections notated using this system, the majority of musicians choose to store music physically, using filing cabinets and music cabinets as storage mechanisms\parencite{musicOrganising}. This can be attributed to the lack of standardisation for browsing and organising digital sheet music. PDF is the standard digitisation method for documents, which presents a problem for musicians wanting to search by multiple methods because the format does not include meta information about what the document contains,\parencite{MusicXMLPresentation} as well as the usability of PDF browsers in musical performances. Whilst the latter has largely been solved by tablet applications\parencite{forScore}, organisation, searching and filtering PDF files is still largely a manual task, with most applications allowing for manual input, but with little to none handling automatic information retrieval\parencite{musicReader}. Further alternative software solutions proving this point will be discussed later in the report.

A further problem in digitising music collections is merging online collections with offline collections, which is not improved by many online and offline music retailers providing only physical copies of compositions\parencite{MusicRoom}. In other words, as collections of music grow, it becomes a long and arduous task to convert physical documents to digital files, and from there to manually attribute meta data for each and every piece.

The physical method avoids this problem, but is still fraught with problems of manual organisation, made more difficult by the need to index or else duplicate files in order to organise collections by multiple methods. This problem is due to music cabinets only allowing the user to view the top document, as well as sheet music books often only providing bibliography information on the front cover, meaning the spine of books does not tell a user what the book contains\parencite{SheetMusicRant}.

This is further complicated by specific use cases, such as music cabinets used by multiple musicians who each play several different instruments. In a solo use case, it may be possible to organise by title, instrument, or composer, but in the given use case finding one which unifies all users needs is difficult.

A further example would be a musical director for an orchestra, who wishes to not only find a specific piece, but find compositions which would work well in a concert schedule. In this case it would be of use to know more than the bibliography of a piece, but information such as time, speed and instruments information without having to physically look or memorise each piece's information. For this reason many larger Orchestras have a dedicated Orchestral Librarian\parencite{MusicLibrarian}, who will handle manual organisation, maintenance and research of the library. This indicates that library sizes take a lot of maintenance, and manual conversion and extraction of meta data to convert a physical library to a digital collection would take a long time.

As such, this project solves the organisation problem by extracting meta data about sheet music automatically, with additional features provided in order to improve the usability and shorten the amount of time needed to create and expand a digital collection of music.
This document discusses the aims and objectives of this project, technical scope and depth of the project, process and method used to produce the solution, and finally critical evaluation of the project as a whole.
\pagebreak
\section{Aims and Objectives}
\subsection{Project Aim}
\begin{center}
\textit{The overall aim of the project is to design and develop a sheet music library application, with the ability to organise and view personal sheet music collections, and download sheet music from the internet. Time permitting, it should also be able to generate sound from the sheet music, provide a difficulty rating for each piece in the catalog and import knowledgeable sheet music from flat images.}
\end{center}
\subsection{Primary Objectives}
The following objectives are of the highest importance to the project, and are a measure of whether the project has been completed.
\begin{itemize}
    \item \textbf{Rendering of Musical Files}\\
    The project successfully produces sheet music from a knowledgeable music file. This portion of the project is necessary as many of the current sheet music viewers are closed source, and as such standard methods for converting from an input knowledgeable file format to an output format are not common.
    \item \textbf{Extraction of Metadata}\\
    The project extracts important information from each piece, ranging from the simple nominal data such as title, composer, lyricist, to the more complex notation such as clefs, key signatures and meters used. 
    \item \textbf{Ability to search metadata extracted and auto-generate playlists}\\ From the extracted metadata it is possible to search the catalog of music for specific requirements, such as key, clef, meter, time signature (explained in the background section). The system also generates automatic playlists based on related data.
    \item \textbf{Connection to Online Music Collections}\\
    The project allows users to expand their local collection by implementing search of online collections using the same methods as searching the local collection. This enables the expansion of local collections without the necessity of using a search engine or browser.

\end{itemize}
\subsection{Secondary Objectives}
The following secondary objectives were defined in order to provide further features to the solution, but with the proviso that the primary objectives were of main concern of the project. As such, these objectives are nice to have, but do not define the success or failure of the project as a whole.

\begin{itemize}
    \item \textbf{Audio playback}\\
    It would be useful to a cross section of users to be able to play music files as sound clips. This enables performers who regularly play with an accompany musician or ensemble to create practice accompaniments from their sheet music, or hear an approximation of how a melody should sound.
    \item \textbf{MusicOCR conversion of images to parseable Music files}\\
It would be easier for musicians to merge their physical and virtual music collections for automatic organisation if the solution provided a way to import flat, scanned sheet music into marked-up music files. 
	\item \textbf{Difficulty rating of each piece based on metadata}\\It would be useful for the system to rate each piece based on the data extracted. This rating is something that most musicians will do upon reading a new piece, and provides more complexity to the project because this is sometimes specific to each instrument and as such will require research into how musicians rate pieces.
\end{itemize}
