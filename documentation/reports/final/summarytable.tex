\subsection{Summary}
Table \ref{table:decisions} gives a summary of the decisions laid out in the above subsections.

\begin{longtable} {| p{.18\textwidth} | p{.18\textwidth} | p{.18\textwidth} | p{.18\textwidth} | p{.18\textwidth} |}  \hline
	{Section} & {Subsection}  & {Options} & {Decision} & {Reason} \\ \hline
	3.2 Technologies & 3.2.1 Programming Language & Python & Python & Easy to use syntax \\
	& & C++ & & Dynamic typing \\
	& & & & \\
	& & C\# & & Larger pool of potential open source developers \\ 
	& & & & \\
	& & & & Cross platform \\ \hline
	 & 3.2.2 File Format & Own Format & MusicXML & Used in several composition applications \\
	 & & & & \\
	& & SIB & & Open format \\
	& & & & \\
	& & MUSCX & & Technical challenge learning new format \\
	& & MusicXML & & \\ \hline
	3.3 Technologies for Rendering & 3.3.1 XML verification & Verifying & Non-Verifying & Faster due to no internet needed \\
	& & & & \\
	& & Non-Verifying & & Files generally automatically created by software which should create valid files \\ \hline
	& 3.3.2 Libraries for XML parsing & DOM & SAX & Better memory management as info not loaded all at once \\
	& & & & \\
	& & SAX & & Easier to use and iteratively build up functionality \\ \hline
	& 3.3.3 Algorithms for display and storage of XML & XSL & Object hierarchy & Extensible: future I/O only needs to create converter to/from objects \\
	& & & & \\
	& & Converter script to output & & O(n) speed result \\
	& & Converter script to object hierarchy & & Avoids coupling to I/O \\ \hline
	& 3.3.4 Rendering System & new system converting objects to render window & output to file to run in Lilypond & Higher quality product as more developers have worked to produce it\\
	& & & & \\
	& & system converting objects to rendered image, then display image & & Avoids duplication of previous research\\
	& & & & more precise automated testing through comparison to expected text output for each class \\
	& & convert objects to Lilypond script & & Technical challenge of learning new format/language \\ \hline
	3.4 Technologies for Organising Sheet Music & 3.4.1 XML data acquisition algorithm & DOM & SAX & better memory management \\
	& & & & \\
	& & SAX & & Iterative build up of functionality \\ \hline
	& & Verifying & Non-Verifying & No internet connection needed \\
	& & & & \\
	& & Non-Verifying & & Files automatically created by trusted composition software, should be valid files. \\ \hline
	& 3.4.2 Metadata Model & Specific to instruments & General to all instruments & Standard data set to all instruments, no special filtering required \\
	& & & & \\
	& & General to all instruments & & Does not rely on language or spelling of instrument names \\
	& & & & \\
	& & & & Provides information which is useful to most musicians \\ \hline
	& 3.4.3 Memory Considerations & In memory object, output to serialised Python file & SQL-based database & Avoids research duplication \\
	& & & & Ensures files do not have to be repeatedly scanned if they haven't changed \\
	& & & & \\
	& & & & Standard database developed by multiple other developers: higher quality + probably faster than memory object developed by 1 developer \\ \hline
	& 3.4.4 Query Processing Considerations & Define new querying syntax all searches must use & Amalgamation of normal input + querying syntax & Simple for simple searches \\
	& & & & \\
	& & Manipulate normal user input to predict type of data required & & Provides more complexity where necessary \\
	& & & & \\
	& & Amalgamation of normal input and querying syntax & & More intuitive than learning a long list of commands for one simple query \\ \hline
	3.5 Technologies for Importing Online Musical Sources & 3.5.2 Searching Algorithm & Search API only, using filters available at API level & Download all files on source, parse for data, combine with data from API and delete file, then download PDF when user asks for it & same quality of data as local files \\
	& & Download all files on source, parse for data, combine with data from API and delete file, then download PDF when user asks for it & & Requires only 2 network calls: at update time and at download time \\ \hline
	& 3.5.3 Licensing Considerations & Only view files with lowest license/no license level & Allow user to "accept" terms of license then download & Large set of data but still covers licensing issue \\
	& & Allow user to "accept" terms of license then download & & \\ \hline
	3.6 Technologies for Sound Output and Image Input & 3.6.2 Optical Music Recognition systems & Create new algorithm & Implement third party system & Avoids research duplication \\
	& & Implement third party system & & Technical challenge of implementing other people's work \\ \hline
	3.7 Technologies for Difficulty Grading & 3.7.1 Metadata Model Considerations & Expand to include symbols which are difficult at general level & General level & useful to all users \\
	& & Expand model to include info specific to instruments & & No issues with language or spelling of instrument names \\ \hline
	& 3.7.2 Rating Algorithm & Do calculation at the time the file is ran through the metadata scanner, assess each symbol in context & Do calculation at the time the file is ran through the metadata scanner, assess each symbol in context & Considered in context \\
	& & & & No bulk collection of data - only collect what is relevant\\
	& & Collect bulk data, compare when metadata scan has finished & &  \\ \hline
	& 3.7.3 Machine Learning considerations & New algorithm/system created by project & Implement third party system & Avoid research duplication \\
	& & & & \\
	& & Implement third party system & & Research area too big to complete in the given time frame as a feature of a larger project \\ \hline
	\caption{Table summarising decisions taken throughout the background section}
\label{table:decisions}
\end{longtable}
