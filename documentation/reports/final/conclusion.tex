\section{Conclusion}
This project intended to solve the problem of organising sheet music. Sheet music is a visual representation of a piece of music which is given to performers in order to understand how a piece of music is to be performed. The organisation problem has many facets to how it could and should be solved, as some useful information about a piece of music is symbolic and some is bibliographic, some information is general, and some information is specific to each instrument \parencite{MIR}. Automating this process is a feature that previous systems have avoided because the information can be subjective and specific to the context in which the information is needed.

Musicians generally use physical storage methods for sheet music because previous systems did not implement automation of data extraction, and thus whether physical or virtual, the user would have needed to create or use a manual tagging system \parencite{musicOrganising}.

This project is considered a success because it alleviates the manual organisation problem by automatically extracting useful information about pieces of music. It uses this information to provide a platform with different options for viewing and searching collections of sheet music as shown and explained in the user guide, and also implements the ability for users to expand their collection using online sources of music by searching using the same interface as they would their local collections.

An orchestral director could use this project to organise arrangements they have produced using composition software, and expand it by searching for pieces online using the application. Similarly, a composer could use it to organise and showcase the pieces they have created, and more easily find pieces they may have produced in the past without needing to remember the file name. 

Whilst the goals of this project were met, the developer learned that the process of rendering sheet music is a more complex process than first expected. The degree of complexity of sheet music notation means that the project does not cover every symbol possible in music composition, but covers enough of the most commonly used symbols to be considered a success, such as dynamics, key signatures, clefs, and a full range of instruments which have parts written for them using the Western Classical notation system.

The developer benefited from having measurable, achievable primary objectives, but with much scope defined by the secondary objectives. The use of test driven development in this project improved the quality of the software and ensured that late design changes did not introduce problems to implemented features.

In the future, it is hoped that more will be done on this project to improve the process of digitising music collections as detailed in the evaluation section. In particular, conversion from images to the format used in this project would make the process from scanning physical copies of sheet music to organising the collection completely automated. Automation of the entire process would mean that the orchestral director would be able to combine their physical collection of music with their virtual collection, reducing space needed to store the physical music library and the time taken to find pieces of music.

The contributions of this project to the field of music information retrieval and organisation are threefold. The first is a rendering system which could be extended and modified depending on new file input and output as decided by future developers, but which is designed to work well with the input and output sources which are currently integrated. The second is a metadata model which is general enough to cover a wide variety of musicians, but with enough symbolic information that it is of more use than the bibliographic and textual data that other systems provide. The third is the project application itself, which provides a usable, extensible application which could integrate new research on a platform which unites the software development community with the music community.