\section{Conclusion}
This project intended to solve the problem of organising sheet music. Sheet music is a visual representation of a piece of music which is given to performers in order to understand how a piece of music is to be performed. The organisation problem has many facets to how it could and should be solved, as some useful information about a piece of music is symbolic and some is bibliographic, some information is general, and some information is specific to each instrument. Automating this process is a feature that previous systems have avoided because the information can be subjective and specific to the context in which the information is needed.

The objectives of having a method to view, organise and expand the collection of music were all met as explained in the evaluation, and therefore this project is considered to be a success. Having an automated information extraction mechanism removes a layer of manual entry when digitising music collections, which should help to move the classical music performance industry into the 21st century. 

Whilst the goals of this project were met, the developer learned that the process of rendering sheet music is a longer process than first expected. The degree of complexity of sheet music notation means that the project does not cover every symbol possible in music composition, but covers enough of the most commonly used symbols to be considered a success.

In the future, it is hoped that more will be done on this project to improve the process of digitising music collections. In particular, conversion from images to the format used in this project would make the process from scanning physical copies of sheet music to organising the collection completely automated. There is further work to be done in improving the rendering capabilities, completing the other secondary objectives and producing an application which works on all platforms, but optical music recognition would be the feature of most use to musicians using this application.


The contributions of this project to the field of music information retrieval and organisation are three fold. The first is a rendering system which could be extended and modified depending on new file input and output as decided by future developers, but which is designed to work well with the input and output sources which are currently integrated. The second is a metadata model which is general enough to cover a wide variety of musicians, but with enough symbolic information that it is of more use than the bibliographic and textual data that other systems provide. The third is the project application itself, which provides a usable, extendable application which could integrate new research on a platform which is designed for musicians, not researchers. The platform could be considered a uniting piece which joins the software development community with the music community.
