\section{Introduction}
This project focuses on the organisation and display of sheet music. Sheet music refers only to the instructions given to a performer in order to play a composition, and does not include the sound output produced when the piece is performed. Any reference to music from this point onward should be assumed to mean visual sheet music, rather than audio recordings.
Whilst Eastern countries and previous eras have used different methods of sheet music notation \parencite{Kaufman}, this project focuses solely on the notation used by western classical music. This type of musical notation is the format most commonly used by orchestras and performers, and in order to understand the problem of organising it, some explanation of the key elements will be required.

Notation of western classical music has used a combination of diastematic and orthographic notation \parencite{RRastall} since the advent of Gregorian chant around 640AD \parencite{RTaruskin}. The function of this notation falls into two main divisions: the expression of relationship in sound frequency, and the expression of relationship in time, or measure \parencite{oxHistory}.

The representation of pitch or sound frequency of a given note is diastematic, meaning that it's relative "highness" or "lowness" in notation provides a visual representation of where it's frequency lies. The expression of relationship in time or measure is given by where, in terms of horizontal or measure position, the particular note falls. Further explanation of these symbols and the meaning of notes and pitches will be explained in the problem context portion of this report.

The representations of other parameters in staff notation are normally orthographic, such as indications p - meaning piano, or "quiet" - and pizz - meaning pizzicato, or "plucked" \parencite{RRastall}. This mechanism is complex in nature, and has a large but finite set of symbols which control every element of the composition. This will be discussed and explained later in this report.

Despite the advantages of digitising music collections notated using this system, the majority of musicians choose to store music physically, using filing cabinets and music cabinets as storage mechanisms \parencite{musicOrganising}. This behaviour can be attributed to the lack of standardisation for browsing and organising digital sheet music. The portable document format (PDF) is the standard digitisation method for documents, which presents a problem for musicians wanting to search by multiple methods because the format does not include meta information about what the document contains \parencite{MusicXMLPresentation}. Another problem in using PDF as a sheet music format is the usability of PDF browsers in musical performances. Whilst the usability problem has largely been solved by tablet applications such as those produced by \cite{forScore}, organisation, searching and filtering PDF files is still largely a manual task. Most applications allow for manual input, but with little to no handling automatic information retrieval \parencite{musicReader}. Further alternative software solutions proving this point will be discussed later in the report.

An example beneficiary of an automated solution would be a musical director for an orchestra, who wishes to not only find a specific piece, but find compositions which would work well in a concert schedule. In this case it would be of use to know not only the bibliography of a piece, but also information such as time, speed and instruments, without having to physically look or memorise each piece. At present, many larger Orchestras have a dedicated orchestral librarian, who will handle manual organisation, maintenance and research of the library, as explained and supported by \cite{MusicLibrarian}. The existence and necessity of having someone employed in this position indicates that large libraries take a lot of maintenance, and manual conversion and extraction of meta data to convert a physical library to a digital collection would take a long time.

As such, this project solves an organisation problem by extracting meta data about sheet music automatically, with additional features provided in order to improve the usability and shorten the amount of time needed to create and expand a digital collection of music.

This document discusses the aims and objectives of this project, technical scope and depth of the project, process and method used to produce the solution, and finally a critical evaluation of the project as a whole.
\pagebreak
\section{Aims and Objectives}
\subsection{Project Aim}
\begin{center}
\textit{The overall aim of the project is to design and develop a sheet music library application, with the ability to organise and view personal sheet music collections, and download sheet music from the internet. Time permitting, it should also be able to generate sound from the sheet music, and import editable music from flat images.}
\end{center}
\subsection{Primary Objectives}
The following objectives are of the highest importance to the project, and are a measure of whether the project has been completed. The technical terms used in this section will be explained in more detail in the problem context.
\begin{itemize}
	\item \textbf{Rendering of sheet music files}\\
    It will be necessary to render one or more formats of commonly stored sheet music files, as the aim of the project is to enable users to \textit{view} and organise their sheet music collections. 
    \item \textbf{Extraction of metadata}\\
    The project will be required to extract important information from each piece, ranging from the simple nominal data such as title, composer, lyricist, to the more complex notation such as clefs, key signatures and meters used. Automatic extraction of information is considered the most important objective of this project, as this is something that is more often a manual job and will be the most useful to users. 
    \item \textbf{Ability to search metadata extracted and auto-generate playlists}\\ From the extracted metadata it should be possible to search the catalog of music for specific requirements, such as key, clef, meter, time signature. The system should also be capable of generating playlists based on related data, in order to provide what the user might want to know without having to search the database.
    \item \textbf{Connection to online music collections}\\
    It should also be possible to connect to online music collections, as it would be beneficial to users to be able to search and add to their collections using the same interface.

\end{itemize}
\subsection{Secondary Objectives}
The secondary objectives are to be completed only if they do not threaten the completion of primary objectives.

\begin{itemize}
    \item \textbf{Audio playback}\\
    It would be useful to a cross section of users to be able to play music files as sound clips. This enables performers who regularly play with an accompany musician or ensemble to create practice accompaniments from their sheet music, or hear an approximation of how a melody should sound.
    \item \textbf{MusicOCR conversion of images to parseable Music files}\\
It would be easier for musicians to merge their physical and virtual music collections for automatic organisation if the solution provided a way to import flat, scanned sheet music into marked-up music files. This is not a primary objective because whilst it would be a useful feature, the research area for MusicOCR is vast and considering it vital to the project may detriment some other more useful and more innovative features. 
	\item \textbf{Difficulty Rating}\\
	In addition to metadata scanning, it would be useful if the system would determine a rating based on the information given of how difficult the piece will be for a performer. This information would help the user to avoid needing to visually scan the music for an indication of how much practice time the piece will require.
\end{itemize}
